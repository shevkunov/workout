\documentclass{article}
\usepackage[T2A]{fontenc}
\usepackage[utf8]{inputenc}
\usepackage{amsmath}
\usepackage{fancyhdr}
\usepackage[english,russian]{babel}
\usepackage{lastpage} 

\pagestyle{fancy} %очистим хидер на всякий случай
\fancyhead[L]{Задача о рюкзаке} 
\fancyhead[C]{Шевкунов К.С. 594} 
\fancyhead[R]{Страница \thepage{}  из \pageref{LastPage}} 
\fancyfoot{} %футер будет пустой



% \theoremstyle{plain}
\newtheorem{theorem}{Теорема} % reset theorem numbering for each chapter
% \ref{t1}

% \theoremstyle{definition}
\newtheorem{definition}{Определение} % definition numbers are dependent on theorem numbers
\newtheorem{example}{Пример}


\usepackage{comment}


\title{Задача о рюкзаке}
\author{Шевкунов К.С. 594}


\begin{document}


\maketitle

\begin{comment}
\section{Постановка задачи}
	\subsection{Формулировка условия}
	\subsection{Цель}
	\subsection{Доказательство NP-полноты}

\section{Псевдополиномиальное решение}
	\subsection{Алгоритм}
	\subsection{Доказательство}
		 
\section{Полиномиальное приближение}
	\subsection{Алгоритм}
	\subsection{Доказательство}
	\subsection{Работа на реальных данных}
\end{comment}



\section{Постановка задачи}

	\subsection{Формулировка условия}

% TODO : переписать условие, чтобы не было мучительно больно 

Имеется набор из $n$ предметов. У каждого предмета есть положительный вес $w$ и стоимость $c$. Также дано неотрицательное число $W$ — вместимость рюкзака.
Требуется найти такое подмножество предметов $M$, чтобы оно помещалось в рюкзак, и суммарная стоимость предметов была максимальна. То есть:

$$\sum\limits_{x \in M} {w(x)} \leq W, \sum\limits_{x \in M} {c(x)} \rightarrow max$$ 

	\subsection{Цель}

Постройте полиномиальную схему приближения для данной задачи. То есть необходимо придумать и реализовать алгоритм, который получает на вход экземпляр задачи о рюкзаке, а также произвольное (рациональное) $\varepsilon > 0$, и находит $(1 + \varepsilon)$ - приближенное решение. Алгоритм должен работать за полиномиальное время относительно размера исходной задачи и $\frac{1}{\varepsilon}$.

	\subsection{Доказательство NP-трудности}
 
Определим задачу: $$SUBSET-SUM = \{(n_1 , . . . , n_k , N ) | \exists \alpha \in \{0, 1\}^k : \sum\limits_{i = 1}^{k} {\alpha_i n_i} = N \}$$  

В книге (Д.В. Мусатов, "Сложность вычислений. Конспект лекций") дока
Конспект лекций) доказано, что задача $SUBSET-SUM$ является NP-полной. Сведём эту задачу полиномиальной к нашей и этим докажем, что она является NP-трудной. 
 
Обозначим предметы натуральными числами $1 .. N$. Определим $\forall i \in \{1..N\} : c(i) := w(i) := n_i$, $W := N$ Тогда исходная задача свелась к поиску $M$ такого, что:

$$\sum\limits_{x \in M} {n_i} \leq N, \sum\limits_{x \in M} {n_i} \rightarrow max$$ 
 
Или, иначе говоря к задаче поиска $\alpha \in \{0, 1\}^k$, такого, что:

$$\sum\limits_{i = 1}^{k} {\alpha_i n_i} \leq N, \sum\limits_{i = 1}^{k} {\alpha_i n_i} \rightarrow max \leq N$$ 

Ясно, что искомый максимум не больше $N$, и, если равняется N, то $(n_1 , . . . , n_k , N ) \in SUBSET-SUM$. Иначе же оптимального подмножества не существует и $(n_1 , . . . , n_k , N ) \not\in SUBSET-SUM$.

Таким образом, с помощью полиномиального сведения мы научились решать задачу $SUBSET-SUM$ c помощью нашей задачи, т.е. с некоторым полиномиальным сведением мы можем решать любую задачу из класса $NP$ с помощью нашей.

Таким образом, мы обосновали уместность рассмотрения приближённых решений для данной задачи. Далее рассмотрим так называемое псевдополиномиальное решение. 


\section{Псевдополиномиальное решение}

	% TODO : сделать определение читаемым
	\begin{definition}
		Пусть в постановку задачи входит числовой параметр n (не количественный) и алгоритм работает полиномиальное время от самого n. Тогда такой алгоритм называется псевдополиномиальным.
	\end{definition}
	В частности, для текущей задачи не известно полиномиального решения (т.е. такого, которое работает за полиномиальное от размера входа время, в частности, для числового параметра N размером входа будет $\log{N}$ ), но известно псевдополиномиальное решение.
	
	\subsection{Алгоритм}
	
	
	
	\subsection{Доказательство}



\section{Полиномиальное приближение}
	\subsection{Алгоритм}
	\subsection{Доказательство}
	\subsection{Работа на реальных данных}
	
\end{document}